% Options for packages loaded elsewhere
\PassOptionsToPackage{unicode}{hyperref}
\PassOptionsToPackage{hyphens}{url}
%
\documentclass[
]{article}
\usepackage{amsmath,amssymb}
\usepackage{iftex}
\ifPDFTeX
  \usepackage[T1]{fontenc}
  \usepackage[utf8]{inputenc}
  \usepackage{textcomp} % provide euro and other symbols
\else % if luatex or xetex
  \usepackage{unicode-math} % this also loads fontspec
  \defaultfontfeatures{Scale=MatchLowercase}
  \defaultfontfeatures[\rmfamily]{Ligatures=TeX,Scale=1}
\fi
\usepackage{lmodern}
\ifPDFTeX\else
  % xetex/luatex font selection
\fi
% Use upquote if available, for straight quotes in verbatim environments
\IfFileExists{upquote.sty}{\usepackage{upquote}}{}
\IfFileExists{microtype.sty}{% use microtype if available
  \usepackage[]{microtype}
  \UseMicrotypeSet[protrusion]{basicmath} % disable protrusion for tt fonts
}{}
\makeatletter
\@ifundefined{KOMAClassName}{% if non-KOMA class
  \IfFileExists{parskip.sty}{%
    \usepackage{parskip}
  }{% else
    \setlength{\parindent}{0pt}
    \setlength{\parskip}{6pt plus 2pt minus 1pt}}
}{% if KOMA class
  \KOMAoptions{parskip=half}}
\makeatother
\usepackage{xcolor}
\usepackage[margin=1in]{geometry}
\usepackage{color}
\usepackage{fancyvrb}
\newcommand{\VerbBar}{|}
\newcommand{\VERB}{\Verb[commandchars=\\\{\}]}
\DefineVerbatimEnvironment{Highlighting}{Verbatim}{commandchars=\\\{\}}
% Add ',fontsize=\small' for more characters per line
\usepackage{framed}
\definecolor{shadecolor}{RGB}{248,248,248}
\newenvironment{Shaded}{\begin{snugshade}}{\end{snugshade}}
\newcommand{\AlertTok}[1]{\textcolor[rgb]{0.94,0.16,0.16}{#1}}
\newcommand{\AnnotationTok}[1]{\textcolor[rgb]{0.56,0.35,0.01}{\textbf{\textit{#1}}}}
\newcommand{\AttributeTok}[1]{\textcolor[rgb]{0.13,0.29,0.53}{#1}}
\newcommand{\BaseNTok}[1]{\textcolor[rgb]{0.00,0.00,0.81}{#1}}
\newcommand{\BuiltInTok}[1]{#1}
\newcommand{\CharTok}[1]{\textcolor[rgb]{0.31,0.60,0.02}{#1}}
\newcommand{\CommentTok}[1]{\textcolor[rgb]{0.56,0.35,0.01}{\textit{#1}}}
\newcommand{\CommentVarTok}[1]{\textcolor[rgb]{0.56,0.35,0.01}{\textbf{\textit{#1}}}}
\newcommand{\ConstantTok}[1]{\textcolor[rgb]{0.56,0.35,0.01}{#1}}
\newcommand{\ControlFlowTok}[1]{\textcolor[rgb]{0.13,0.29,0.53}{\textbf{#1}}}
\newcommand{\DataTypeTok}[1]{\textcolor[rgb]{0.13,0.29,0.53}{#1}}
\newcommand{\DecValTok}[1]{\textcolor[rgb]{0.00,0.00,0.81}{#1}}
\newcommand{\DocumentationTok}[1]{\textcolor[rgb]{0.56,0.35,0.01}{\textbf{\textit{#1}}}}
\newcommand{\ErrorTok}[1]{\textcolor[rgb]{0.64,0.00,0.00}{\textbf{#1}}}
\newcommand{\ExtensionTok}[1]{#1}
\newcommand{\FloatTok}[1]{\textcolor[rgb]{0.00,0.00,0.81}{#1}}
\newcommand{\FunctionTok}[1]{\textcolor[rgb]{0.13,0.29,0.53}{\textbf{#1}}}
\newcommand{\ImportTok}[1]{#1}
\newcommand{\InformationTok}[1]{\textcolor[rgb]{0.56,0.35,0.01}{\textbf{\textit{#1}}}}
\newcommand{\KeywordTok}[1]{\textcolor[rgb]{0.13,0.29,0.53}{\textbf{#1}}}
\newcommand{\NormalTok}[1]{#1}
\newcommand{\OperatorTok}[1]{\textcolor[rgb]{0.81,0.36,0.00}{\textbf{#1}}}
\newcommand{\OtherTok}[1]{\textcolor[rgb]{0.56,0.35,0.01}{#1}}
\newcommand{\PreprocessorTok}[1]{\textcolor[rgb]{0.56,0.35,0.01}{\textit{#1}}}
\newcommand{\RegionMarkerTok}[1]{#1}
\newcommand{\SpecialCharTok}[1]{\textcolor[rgb]{0.81,0.36,0.00}{\textbf{#1}}}
\newcommand{\SpecialStringTok}[1]{\textcolor[rgb]{0.31,0.60,0.02}{#1}}
\newcommand{\StringTok}[1]{\textcolor[rgb]{0.31,0.60,0.02}{#1}}
\newcommand{\VariableTok}[1]{\textcolor[rgb]{0.00,0.00,0.00}{#1}}
\newcommand{\VerbatimStringTok}[1]{\textcolor[rgb]{0.31,0.60,0.02}{#1}}
\newcommand{\WarningTok}[1]{\textcolor[rgb]{0.56,0.35,0.01}{\textbf{\textit{#1}}}}
\usepackage{graphicx}
\makeatletter
\def\maxwidth{\ifdim\Gin@nat@width>\linewidth\linewidth\else\Gin@nat@width\fi}
\def\maxheight{\ifdim\Gin@nat@height>\textheight\textheight\else\Gin@nat@height\fi}
\makeatother
% Scale images if necessary, so that they will not overflow the page
% margins by default, and it is still possible to overwrite the defaults
% using explicit options in \includegraphics[width, height, ...]{}
\setkeys{Gin}{width=\maxwidth,height=\maxheight,keepaspectratio}
% Set default figure placement to htbp
\makeatletter
\def\fps@figure{htbp}
\makeatother
\setlength{\emergencystretch}{3em} % prevent overfull lines
\providecommand{\tightlist}{%
  \setlength{\itemsep}{0pt}\setlength{\parskip}{0pt}}
\setcounter{secnumdepth}{-\maxdimen} % remove section numbering
\ifLuaTeX
  \usepackage{selnolig}  % disable illegal ligatures
\fi
\usepackage{bookmark}
\IfFileExists{xurl.sty}{\usepackage{xurl}}{} % add URL line breaks if available
\urlstyle{same}
\hypersetup{
  pdftitle={STATS 10 Assignment 5},
  pdfauthor={Your Name},
  hidelinks,
  pdfcreator={LaTeX via pandoc}}

\title{STATS 10 Assignment 5}
\author{Your Name}
\date{2025-03-12}

\begin{document}
\maketitle

\begin{Shaded}
\begin{Highlighting}[]
\NormalTok{knitr}\SpecialCharTok{::}\NormalTok{opts\_chunk}\SpecialCharTok{$}\FunctionTok{set}\NormalTok{(}\AttributeTok{echo =} \ConstantTok{TRUE}\NormalTok{)}
\end{Highlighting}
\end{Shaded}

\#Part I Exercise 1 (a)

\begin{Shaded}
\begin{Highlighting}[]
\NormalTok{pawnee }\OtherTok{\textless{}{-}} \FunctionTok{read.csv}\NormalTok{(}\StringTok{"pawnee.csv"}\NormalTok{, }\AttributeTok{header =} \ConstantTok{TRUE}\NormalTok{)}
\FunctionTok{head}\NormalTok{(pawnee)}
\end{Highlighting}
\end{Shaded}

\begin{verbatim}
##   ID Latitude Longitude Arsenic Sulfur New_hlth_issue
## 1  1 41.09414 -85.60974       0      0              N
## 2  2 41.09054 -85.70344       0    130              N
## 3  3 41.08601 -85.71996       4    170              N
## 4  4 41.08100 -85.75415       0      0              Y
## 5  5 41.07435 -85.70043       0      0              N
## 6  6 41.07399 -85.71788       0      0              N
\end{verbatim}

\begin{Shaded}
\begin{Highlighting}[]
\FunctionTok{dim}\NormalTok{(pawnee)}
\end{Highlighting}
\end{Shaded}

\begin{verbatim}
## [1] 541   6
\end{verbatim}

\begin{enumerate}
\def\labelenumi{(\alph{enumi})}
\setcounter{enumi}{1}
\tightlist
\item
\end{enumerate}

\begin{Shaded}
\begin{Highlighting}[]
\FunctionTok{set.seed}\NormalTok{(}\DecValTok{1337}\NormalTok{)}
\NormalTok{sample\_pawnee }\OtherTok{\textless{}{-}}\NormalTok{ pawnee[}\FunctionTok{sample}\NormalTok{(}\FunctionTok{nrow}\NormalTok{(pawnee), }\DecValTok{30}\NormalTok{), ]}
\FunctionTok{head}\NormalTok{(sample\_pawnee)}
\end{Highlighting}
\end{Shaded}

\begin{verbatim}
##      ID Latitude Longitude Arsenic Sulfur New_hlth_issue
## 147 147 41.03971 -85.72783       2    100              N
## 49   49 41.06113 -85.65553       0      0              Y
## 210 210 41.03178 -85.64253       0      0              N
## 356 356 41.01178 -85.66516       0      0              N
## 425 425 41.00096 -85.72899       0      0              N
## 239 239 41.02772 -85.72901       0      0              N
\end{verbatim}

\begin{enumerate}
\def\labelenumi{(\alph{enumi})}
\setcounter{enumi}{2}
\tightlist
\item
\end{enumerate}

\begin{Shaded}
\begin{Highlighting}[]
\NormalTok{prop\_sample }\OtherTok{\textless{}{-}} \FunctionTok{mean}\NormalTok{(sample\_pawnee}\SpecialCharTok{$}\NormalTok{New\_hlth\_issue }\SpecialCharTok{==} \StringTok{"Y"}\NormalTok{)}
\NormalTok{prop\_population }\OtherTok{\textless{}{-}} \FunctionTok{mean}\NormalTok{(pawnee}\SpecialCharTok{$}\NormalTok{New\_hlth\_issue }\SpecialCharTok{==} \StringTok{"Y"}\NormalTok{)}
\NormalTok{prop\_sample}
\end{Highlighting}
\end{Shaded}

\begin{verbatim}
## [1] 0.2
\end{verbatim}

\begin{Shaded}
\begin{Highlighting}[]
\NormalTok{prop\_population}
\end{Highlighting}
\end{Shaded}

\begin{verbatim}
## [1] 0.2920518
\end{verbatim}

\begin{enumerate}
\def\labelenumi{(\alph{enumi})}
\setcounter{enumi}{3}
\tightlist
\item
\end{enumerate}

\begin{Shaded}
\begin{Highlighting}[]
\NormalTok{n }\OtherTok{\textless{}{-}} \DecValTok{30}
\NormalTok{p\_hat }\OtherTok{\textless{}{-}}\NormalTok{ prop\_sample}
\NormalTok{se }\OtherTok{\textless{}{-}} \FunctionTok{sqrt}\NormalTok{(p\_hat }\SpecialCharTok{*}\NormalTok{ (}\DecValTok{1} \SpecialCharTok{{-}}\NormalTok{ p\_hat) }\SpecialCharTok{/}\NormalTok{ n)}

\NormalTok{z\_90 }\OtherTok{\textless{}{-}} \FunctionTok{qnorm}\NormalTok{(}\FloatTok{0.95}\NormalTok{)}
\NormalTok{ci90 }\OtherTok{\textless{}{-}} \FunctionTok{c}\NormalTok{(p\_hat }\SpecialCharTok{{-}}\NormalTok{ z\_90 }\SpecialCharTok{*}\NormalTok{ se, p\_hat }\SpecialCharTok{+}\NormalTok{ z\_90 }\SpecialCharTok{*}\NormalTok{ se)}

\NormalTok{z\_95 }\OtherTok{\textless{}{-}} \FunctionTok{qnorm}\NormalTok{(}\FloatTok{0.975}\NormalTok{)}
\NormalTok{ci95 }\OtherTok{\textless{}{-}} \FunctionTok{c}\NormalTok{(p\_hat }\SpecialCharTok{{-}}\NormalTok{ z\_95 }\SpecialCharTok{*}\NormalTok{ se, p\_hat }\SpecialCharTok{+}\NormalTok{ z\_95 }\SpecialCharTok{*}\NormalTok{ se)}

\NormalTok{z\_99 }\OtherTok{\textless{}{-}} \FunctionTok{qnorm}\NormalTok{(}\FloatTok{0.995}\NormalTok{)}
\NormalTok{ci99 }\OtherTok{\textless{}{-}} \FunctionTok{c}\NormalTok{(p\_hat }\SpecialCharTok{{-}}\NormalTok{ z\_99 }\SpecialCharTok{*}\NormalTok{ se, p\_hat }\SpecialCharTok{+}\NormalTok{ z\_99 }\SpecialCharTok{*}\NormalTok{ se)}

\NormalTok{ci90}
\end{Highlighting}
\end{Shaded}

\begin{verbatim}
## [1] 0.07987688 0.32012312
\end{verbatim}

\begin{Shaded}
\begin{Highlighting}[]
\NormalTok{ci95}
\end{Highlighting}
\end{Shaded}

\begin{verbatim}
## [1] 0.05686447 0.34313553
\end{verbatim}

\begin{Shaded}
\begin{Highlighting}[]
\NormalTok{ci99}
\end{Highlighting}
\end{Shaded}

\begin{verbatim}
## [1] 0.01188802 0.38811198
\end{verbatim}

Exercise 2. (a) Answer: Null hypothesis (H₀): The proportion of
dangerous lead levels is less than or equal to 10\% (i.e.,
p\textless=0.10). Alternative hypothesis (Hₐ): The proportion is greater
than 10\% (i.e., p\textgreater0.10) This is a one-sided test.

\begin{enumerate}
\def\labelenumi{(\alph{enumi})}
\setcounter{enumi}{1}
\tightlist
\item
\end{enumerate}

\begin{Shaded}
\begin{Highlighting}[]
\NormalTok{flint }\OtherTok{\textless{}{-}} \FunctionTok{read.csv}\NormalTok{(}\StringTok{"flint.csv"}\NormalTok{, }\AttributeTok{header =} \ConstantTok{TRUE}\NormalTok{)}
\NormalTok{n }\OtherTok{\textless{}{-}} \FunctionTok{dim}\NormalTok{(flint)[}\DecValTok{1}\NormalTok{]}
\NormalTok{p\_null }\OtherTok{\textless{}{-}} \FloatTok{0.1}

\NormalTok{p\_hat }\OtherTok{\textless{}{-}} \FunctionTok{mean}\NormalTok{(flint}\SpecialCharTok{$}\NormalTok{Pb }\SpecialCharTok{\textgreater{}=} \DecValTok{15}\NormalTok{)}
\NormalTok{sample\_sd }\OtherTok{\textless{}{-}} \FunctionTok{sqrt}\NormalTok{(p\_hat }\SpecialCharTok{*}\NormalTok{ (}\DecValTok{1} \SpecialCharTok{{-}}\NormalTok{ p\_hat) }\SpecialCharTok{/}\NormalTok{ n)}
\NormalTok{p\_hat}
\end{Highlighting}
\end{Shaded}

\begin{verbatim}
## [1] 0.04436229
\end{verbatim}

\begin{Shaded}
\begin{Highlighting}[]
\NormalTok{sample\_sd}
\end{Highlighting}
\end{Shaded}

\begin{verbatim}
## [1] 0.008852277
\end{verbatim}

\begin{enumerate}
\def\labelenumi{(\alph{enumi})}
\setcounter{enumi}{2}
\tightlist
\item
\end{enumerate}

\begin{Shaded}
\begin{Highlighting}[]
\CommentTok{\# Standard error using the null hypothesis proportion p0 = 0.10}
\NormalTok{se\_flint }\OtherTok{\textless{}{-}} \FunctionTok{sqrt}\NormalTok{(p\_null }\SpecialCharTok{*}\NormalTok{ (}\DecValTok{1} \SpecialCharTok{{-}}\NormalTok{ p\_null) }\SpecialCharTok{/}\NormalTok{ n)}

\CommentTok{\# Calculate the z{-}value}
\NormalTok{z\_value }\OtherTok{\textless{}{-}}\NormalTok{ (p\_hat }\SpecialCharTok{{-}}\NormalTok{ p\_null) }\SpecialCharTok{/}\NormalTok{ se\_flint}
\NormalTok{z\_value}
\end{Highlighting}
\end{Shaded}

\begin{verbatim}
## [1] -4.313667
\end{verbatim}

\begin{enumerate}
\def\labelenumi{(\alph{enumi})}
\setcounter{enumi}{3}
\tightlist
\item
\end{enumerate}

\begin{Shaded}
\begin{Highlighting}[]
\CommentTok{\# For a one{-}sided test (p \textgreater{} 0.10)}
\NormalTok{p\_value }\OtherTok{\textless{}{-}} \DecValTok{1} \SpecialCharTok{{-}} \FunctionTok{pnorm}\NormalTok{(z\_value)}
\NormalTok{p\_value}
\end{Highlighting}
\end{Shaded}

\begin{verbatim}
## [1] 0.999992
\end{verbatim}

\begin{enumerate}
\def\labelenumi{(\alph{enumi})}
\setcounter{enumi}{4}
\tightlist
\item
  Decision at α = 0.05
\end{enumerate}

\begin{Shaded}
\begin{Highlighting}[]
\NormalTok{p\_value }\SpecialCharTok{\textless{}=} \FloatTok{0.5}
\end{Highlighting}
\end{Shaded}

\begin{verbatim}
## [1] FALSE
\end{verbatim}

Our p-value is larger than 0.05 so we fail to reject the null
hypothesis.

\begin{enumerate}
\def\labelenumi{(\alph{enumi})}
\setcounter{enumi}{5}
\tightlist
\item
  Determining whether to be told to the EPA since remediation action is
  required to be taken by the EPA if greater than 10\% of households in
  Flint contain dangerous lead levels.
\end{enumerate}

Since we failed to reject the null hypothesis, we have no evidence to
suggest that more than 10\% of homes in Flint have a dangerous lead
level. Therefore, we would tell the EPA that they are not required to
take remediation action.

\#Part II

Exercise 1 (a)

\begin{itemize}
\tightlist
\item
  Null Hypothesis (H₀): p = p₀ = 0.48, the proportion of site users who
  get their world news on the site has not changed since 2013.
\item
  Alternative Hypothesis (Hₐ): p ≠ p₀, the proportion of site users who
  get their world news on the site has changed since 2013.
\end{itemize}

The conditions for using the z-test are satisfied since: 1. The sample
is randomly selected 2. n x p₀ = 3625 x 0.48 = 1740 \textgreater{} 10 \&
n(1 - p₀) = 1885 \textgreater{} 10 3. Population N is large as N ≥ 10n =
10 x 3625 = 36250 We calculate the sample proportion (p̂) as 1830/3625 ≈
0.5047, the standard error (SE) as √(p₀ (1-p₀)/n) ≈ 0.0083, and the
z-test statistic as (p̂ - p₀) / SE ≈ 2.99

We calculated a z-score of approximately 2.99. For the two-tailed test
(Hₐ: p ≠ p₀), we assess the p-value = P(Z ≤ -\textbar z\textbar) + P(Z ≥
\textbar z\textbar) = 2(1 -- 0.9986) = 0.0028 The p-value was calculated
to be approximately 0.0028.

We reject the null hypothesis (H₀) since the p-value ≈ 0.0028 is less
than the significance level (α) of 0.05. There is statistical evidence
to suggest that the proportion of site users who get their world news on
the site has changed since 2013.

\begin{enumerate}
\def\labelenumi{(\alph{enumi})}
\setcounter{enumi}{1}
\tightlist
\item
\end{enumerate}

\begin{itemize}
\item
  Calculating 95\% confidence interval: p̂ ± z* x SE = 0.5047 ± 1.96 x
  0.0083 = 0.5047 ± 0.016268 = (0.489, 0.521)
\item
  Interpretation: The confidence interval provides a range of plausible
  values for the population proportion based on the sample data. Since
  this confidence interval does not include the 2013 proportion of 0.48,
  it suggests that there has been a significant change in the proportion
  of site users getting their news from the site since 2013 with the
  true population proportion being higher in 2018. Further, the interval
  suggests that if we were to take many samples of the same size from
  the population, about 95\% of those samples would result in a sample
  proportion between 0.489 and 0.521, further indicating an increase
  from the 2013 proportion. In summary, the confidence interval does not
  contain the 2013 value of 0.48 and lies entirely above it, which is
  consistent with the hypothesis test result indicating a significant
  increase from the 2013 proportion.
\end{itemize}

Exercise 2 --

Type I Error: Concluding that the proportion of voters in 2018 is higher
than 50\% when, in reality, it is 50\% (or not higher). This is a false
positive. Type II Error: Failing to detect that the proportion is higher
when it actually is (i.e., not rejecting H₀ when the true proportion is
above 50\%). This is a false negative.

Exercise 3

\begin{enumerate}
\def\labelenumi{(\alph{enumi})}
\tightlist
\item
\end{enumerate}

\begin{Shaded}
\begin{Highlighting}[]
\CommentTok{\# Data for 2016}
\NormalTok{n1 }\OtherTok{\textless{}{-}} \DecValTok{3103}
\NormalTok{x1 }\OtherTok{\textless{}{-}} \DecValTok{2087}
\NormalTok{p1 }\OtherTok{\textless{}{-}}\NormalTok{ x1 }\SpecialCharTok{/}\NormalTok{ n1}

\CommentTok{\# Data for 2017}
\NormalTok{n2 }\OtherTok{\textless{}{-}} \DecValTok{2988}
\NormalTok{x2 }\OtherTok{\textless{}{-}} \DecValTok{1930}
\NormalTok{p2 }\OtherTok{\textless{}{-}}\NormalTok{ x2 }\SpecialCharTok{/}\NormalTok{ n2}

\CommentTok{\# Pooled proportion (under H0: p1 = p2)}
\NormalTok{p\_pool }\OtherTok{\textless{}{-}}\NormalTok{ (x1 }\SpecialCharTok{+}\NormalTok{ x2) }\SpecialCharTok{/}\NormalTok{ (n1 }\SpecialCharTok{+}\NormalTok{ n2)}

\CommentTok{\# Standard error using the pooled proportion}
\NormalTok{se\_diff }\OtherTok{\textless{}{-}} \FunctionTok{sqrt}\NormalTok{(p\_pool }\SpecialCharTok{*}\NormalTok{ (}\DecValTok{1} \SpecialCharTok{{-}}\NormalTok{ p\_pool) }\SpecialCharTok{*}\NormalTok{ (}\DecValTok{1}\SpecialCharTok{/}\NormalTok{n1 }\SpecialCharTok{+} \DecValTok{1}\SpecialCharTok{/}\NormalTok{n2))}

\CommentTok{\# Calculate the z{-}statistic}
\NormalTok{z\_college }\OtherTok{\textless{}{-}}\NormalTok{ (p1 }\SpecialCharTok{{-}}\NormalTok{ p2) }\SpecialCharTok{/}\NormalTok{ se\_diff}
\NormalTok{z\_college}
\end{Highlighting}
\end{Shaded}

\begin{verbatim}
## [1] 2.194808
\end{verbatim}

\begin{Shaded}
\begin{Highlighting}[]
\CommentTok{\# Two{-}tailed p{-}value}
\NormalTok{p\_value\_college }\OtherTok{\textless{}{-}} \DecValTok{2} \SpecialCharTok{*}\NormalTok{ (}\DecValTok{1} \SpecialCharTok{{-}} \FunctionTok{pnorm}\NormalTok{(}\FunctionTok{abs}\NormalTok{(z\_college)))}
\NormalTok{p\_value\_college}
\end{Highlighting}
\end{Shaded}

\begin{verbatim}
## [1] 0.02817737
\end{verbatim}

Interpretation: Since the p-value is less than 0.05, we conclude there
is a significant change between 2016 and 2017.

\begin{enumerate}
\def\labelenumi{(\alph{enumi})}
\setcounter{enumi}{1}
\tightlist
\item
\end{enumerate}

\begin{Shaded}
\begin{Highlighting}[]
\CommentTok{\# Standard error for the difference (using separate variances)}
\NormalTok{se\_diff\_np }\OtherTok{\textless{}{-}} \FunctionTok{sqrt}\NormalTok{((p1 }\SpecialCharTok{*}\NormalTok{ (}\DecValTok{1} \SpecialCharTok{{-}}\NormalTok{ p1) }\SpecialCharTok{/}\NormalTok{ n1) }\SpecialCharTok{+}\NormalTok{ (p2 }\SpecialCharTok{*}\NormalTok{ (}\DecValTok{1} \SpecialCharTok{{-}}\NormalTok{ p2) }\SpecialCharTok{/}\NormalTok{ n2))}

\CommentTok{\# 95\% Confidence Interval for the difference (p1 {-} p2)}
\NormalTok{ci\_diff\_lower }\OtherTok{\textless{}{-}}\NormalTok{ (p1 }\SpecialCharTok{{-}}\NormalTok{ p2) }\SpecialCharTok{{-}} \FloatTok{1.96} \SpecialCharTok{*}\NormalTok{ se\_diff\_np}
\NormalTok{ci\_diff\_upper }\OtherTok{\textless{}{-}}\NormalTok{ (p1 }\SpecialCharTok{{-}}\NormalTok{ p2) }\SpecialCharTok{+} \FloatTok{1.96} \SpecialCharTok{*}\NormalTok{ se\_diff\_np}
\NormalTok{ci\_diff }\OtherTok{\textless{}{-}} \FunctionTok{c}\NormalTok{(ci\_diff\_lower, ci\_diff\_upper)}
\NormalTok{ci\_diff}
\end{Highlighting}
\end{Shaded}

\begin{verbatim}
## [1] 0.002852873 0.050462979
\end{verbatim}

Since the 95\% confidence interval does not include 0, it supports the
conclusion that there is a significant difference between the two years.

\end{document}
